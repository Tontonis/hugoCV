%______________________________________________________________________________________________________________________
% @brief    LaTeX2e Resume for Kamil K Wojcicki
\documentclass[margin,line]{resume}


%______________________________________________________________________________________________________________________
\begin{document}
\name{\Large Hugo Day}
\begin{resume}

    %__________________________________________________________________________________________________________________
    % Contact Information
    \section{\mysidestyle Contact\\Information}

    CERN                            \hfill office: +61 7 3735 3754          \vspace{0mm}\\\vspace{0mm}%
    Bt. 9-1-002                          \hfill mobile: +44 7890 945140          \vspace{0mm}\\\vspace{0mm}%
    CH-1211, Geneve 23, Switzerland      \hfill e-mail: hugoaday@gmail.com  \vspace{0mm}\\\vspace{-4.5mm}%


    %__________________________________________________________________________________________________________________
    % Research Interests



    %__________________________________________________________________________________________________________________
    % Education
    \section{\mysidestyle Education}

    \textbf{University of Manchester}, Manchester, United Kingdom \vspace{2mm}\\\vspace{1mm}%
    \textsl{Doctor of Philosophy} \hfill \textbf{ September 2009 -- present}\vspace{-3mm}\\\vspace{-1mm}%
    \begin{list2}
        \item Thesis title: ``Measurements and Simulations of Impedance Reduction Techniques in Particle Accelerators" - Work done under placement at CERN (February 2010 -- February 2013) through the CERN Doctoral Program
        \item Expected graduation date: February 2013
        \item Supervisors:  Professor Roger Jones and Dr Elias Metral (CERN)
        \item I used theoretical models, commercial EM modelling software (CST Microwave Studio and Ansoft HFSS) and bench top RF measurements to evaluate and characterise the EM properties of accelerator components, predominantly fast transmission line kickers, in the auspices of beam coupling impedance and heating due to high current beam interactions. Working in an internationally diverse team, results were regularly presented internally for review and feedback, with communication between operational and design teams to integrate the RF results into their design revisions. Durng the course of this thesis a new RF measuring technique for evaluating asymmetric structures was proposed and verified with simulations and measurements, and the power loss in ferrite damped cavities in the case of weak to moderate damping investigated in detail.
    \end{list2}\vspace{-1.5mm}
    \textbf{ University of Southampton}, Southampton, United Kingdom 
    \textsl{MPhys (Hons) Physics Classification: 1$^{\mathrm{st}}$ Class} \hfill \textbf{October 2005 -- July 2009}\vspace{-3mm}\\\vspace{-1mm}%
    \begin{list2}
        \item Masters' Thesis: ``Controlling the synthesis of branched gold nanoparticles using a wet chemical synthesis method'' under the supervision of Dr. Antonios Kanaras
        \item Supervisor: Dr Malcolm Coe
    \end{list2}\vspace{-1.5mm}

    %__________________________________________________________________________________________________________________
    % Professional Experience
    \section{\mysidestyle Professional\\Experience}


    \textbf{CERN}, Geneve, United Kingdom \vspace{2mm}\\\vspace{1mm}%
    \textsl{CERN Tour Guide/Conferencier} \hfill \textbf{November 2010 -- Present}\\
    Giving introductory presentations and tours of experimental facilities at CERN to visitors, both public and private of all ages and knowledge.

    \textbf{University of Manchester}, Manchester, United Kingdom \vspace{2mm}\\\vspace{1mm}%
    \textsl{Teaching Assistant} \hfill \textbf{September 2010 -- January 2011}\\
    Teaching assistant for undergraduate courses (Programming in C).

    \textbf{CERN}, Geneve, United Kingdom \vspace{2mm}\\\vspace{1mm}%
    \textsl{CERN Summer Student} \hfill \textbf{June 2008 -- September 2008}\\
    Simulations of the H$^{-}$ and electrons through the spectrometer in the 3MeV test stand for LINAC4. The magnetic field of the spectrometer was first realised, and then the trajectory of the particles simulated using the tracking code PATH.


    %__________________________________________________________________________________________________________________
    % Peer-Reviewed Publications
    \section{\mysidestyle Peer-Reviewed Publications}

    
	Day, H.A.,  Bartczak, D.,  Fairbairn, N.,  McGuire, E., Ardakan, M., Porter, A. E. and Kanaras, A.G.,
    ``Controlling the three-dimensional morphology of nanocrystals'',
    \textsl{CrystEngComm}, 2010, 12, 4312-4316.

\vspace{-2mm}


    %__________________________________________________________________________________________________________________
    % Conference Publications
    \section{\mysidestyle Conference Publications}

    
	Day, H.A., Barnes, M., Caspers, F., Jones, R.M., Metral, E., Salvant, B.
    ``Evaluation of the Beam Coupling Impedance of New Beam Screen Designs for the LHC Injection Kicker Magnets'',
    WEPPR071, IPAC'12, New Orleans, US, 2012.

\vspace{-2mm}

	Day, H.A., Caspers, F., Dallocchio, A., Gentini, L., Grudiev, A., Jones, R.M., Metral, E., Salvant, B.
    ``Beam Coupling Impedance of the LHC TCTP Collimators'',
    WEPPR070, IPAC'12, New Orleans, US, 2012.

\vspace{-2mm}

	Day, H.A., Caspers, F., Jones, R.M., Metral, E.
    ``Simulations of Coaxial Wire Measurements of the Impedance of Asymmetric Structures. '',
    MOPS079, IPAC'11, San Sebastian, Spain, 2011.

\vspace{-2mm}

	Day, H.A., Barnes, M., Caspers, F., Jones, R.M., Metral, E., Salvant, B., C. Zannini
    ``Coaxial Wire Measurements of Ferrite Kicker Magnets'',
    MOPS078, IPAC'11, San Sebastian, Spain, 2011.

\vspace{-2mm}

	Day, H.A., Caspers, F., Jones, R.M., Metral, E., Salvant, B.
    ``Comparison of the current LHC Collimators and the SLAC Phase 2 Collimator Impedances'',
    MOPS080, IPAC'11, San Sebastian, Spain, 2011.

    %__________________________________________________________________________________________________________________
    % Invited Talks
    \section{\mysidestyle Invited Talks}

	Day, H.A., Biancacci, N., Salvant, B. Zannini, C.
    ``Impedance simulations of the LHC collimators and low beta simulations of ferrite kicker magnets with CST Particle Studio'',
    6th CST European User Group Meeting, Freising, Germany, 2011.


    %__________________________________________________________________________________________________________________
    % Language Skills
    \section{\mysidestyle Languages} 

   English (Fluent), French (Intermediate), German (Beginner)


    %__________________________________________________________________________________________________________________
    % Computer Skills
    \section{\mysidestyle Programming} 

    C, C++, Matlab, Mathematica, Python, Java, \LaTeXe, R, *nix and Windows administration



    %__________________________________________________________________________________________________________________
    % Referees
%    \section{\mysidestyle Referees} 
%    {\sl Available on request.}



%______________________________________________________________________________________________________________________
\section{\mysidestyle Referees} 

\begin{tabular}{@{}p{6cm}p{6cm}}
\textbf{Dr. Elias Metral}       &  \textbf{Dr Roger Jones}                   \\
CERN                     &  University of Manchester                      \\
Geneve 23, Switzerland, CH-1211           &  Oxford Road, Manchester, United Kingdom, M13 9PL        \\
phone: \textsl{available on request}    &  phone: \textsl{available on request}     \\
e-mail: \textsl{elias.metral@cern.ch}   &  e-mail: \textsl{roger.jones@stfc.ac.uk}    \\
\end{tabular}

\begin{tabular}{@{}p{6cm}}
\textbf{Dr Fritz Caspers}           \\
CERN                    		         \\
Geneve 23, Switzerland, CH-1211 \\      
phone: \textsl{available on request} \\  
e-mail: \textsl{fritz.caspers@cern.ch} \\
\end{tabular}



%______________________________________________________________________________________________________________________
\end{resume}
\end{document}


%______________________________________________________________________________________________________________________
% EOF

