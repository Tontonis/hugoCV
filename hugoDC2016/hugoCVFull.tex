%______________________________________________________________________________________________________________________
% @brief    LaTeX2e Resume for Kamil K Wojcicki
\documentclass[margin,line]{resume}


%______________________________________________________________________________________________________________________
\begin{document}
\name{\Large Hugo Day}
\begin{resume}

    %__________________________________________________________________________________________________________________
    % Personal Information
    \section{\mysidestyle Personal Details}
    Nationality: British \vspace{0mm}\\\vspace{-4.5mm}%
    Full UK Driver's Licence  \vspace{0mm}\\\vspace{-4.5mm}%


    %__________________________________________________________________________________________________________________
    % Contact Information
    \section{\mysidestyle Contact\\Information}
    Avenue du Jura, 49                            \hfill mobile: +44 7890 945140          \vspace{0mm}\\\vspace{0mm}%
    Ferney-Voltaire, 01210, France                          \hfill e-mail: hugoaday@gmail.com  \vspace{0mm}\\\vspace{-4.5mm} %
    https://ch.linkedin.com/in/hugo-day-610474112    \hfill https://github.com/Tontonis  \vspace{0mm}\\\vspace{-4.5mm}%

    %__________________________________________________________________________________________________________________
    % Research Interests
%    \section{\mysidestyle Research\\Interests}
%
   \section{\mysidestyle Professional Statement}

    I am an applied physicist with extensive experience working in technically challenging, international environments, mostly in an academic or humanitarian setting. 5+ years experience working with particle accelerators, with an emphasis on electromagnetics and RF engineering. I'm active in the hacker community and am committed to continued professional development and outreach/education in STEM subjects.

%Beam Dynamics, Wakefields and beam impedance, RF engineering, beam instrumentation, complex systems, numerical methods, nanotechnology, solid state physics

    \section{\mysidestyle Proficiencies}

Problem solving, experimental work/design, data analysis, RF engineering, complex systems, numerical methods, technical design and presentation


    %__________________________________________________________________________________________________________________
    % Computer Skills
    \section{\mysidestyle Computing Skills} 

    HFSS, CST: Microwave Studio, pSpice, Cadence, C, C++, Matlab, Mathematica, Python, SciPy, Matplotlib, Java, \LaTeXe, R, Git, *nix and Windows administration, openoffice and MS Office

    %__________________________________________________________________________________________________________________
    % Professional Experience
    \section{\mysidestyle Professional\\Experience}

    \textbf{CERN}, Geneva, Switzerland \vspace{2mm}\\\vspace{1mm}%
    \textsl{Senior Research Fellow} \hfill \textbf{July 2013 -- June 2016}\\
    Worked in the fast pulsed systems group of the technology department on various topics related to the LHC injection kicker magnets, focused on electromagnetic interactions between the particle beam and the electromagnet. Worked extensively with standard EM software (HFSS, CST:MWS, pSpice, Cadence) and bench RF test stands (VNAs).
    \begin{itemize}
    \item{Proposed an upgrade to EM performance of magnet which was subsequently applied to all kicker magnets during 2013 maintenance shutdown}
    \item{Lead study of HV behaviour of cathode triple junction as applied to LHC kicker magnet geometry}
    \item{Implemented an library of analysis tools for measurements and simulations in python for transparency and reuse}
    \end{itemize}

    \textbf{CERN}, Geneva, Switzerland \vspace{2mm}\\\vspace{1mm}%
    \textsl{Doctoral Student} \hfill \textbf{February 2010 -- January 2013}\\
    Examining the electromagnetic interactions between charged particle beams and particle accelerator components using simulation suites and bench-top measurements techniques. Focused on normal-conducting fast pulsed electromagnets and collimators (robust beam protection equipment) necessitating dealing with competing design requirements to find optimal solutions. 
    \begin{itemize}
    \item{Lead simulation and measurement campaign of beam coupling impedance of the LHC injection magnets, collaborating with colleagues in other departments at CERN and in other research institutes in the UK and Germany}
    \item{Devised an analysis technique to measure beam impedance in asymmetric structures}
    \item{Carried out a systematic review of impedance reduction techniques, their advantages/disadvantages and implementation}
    \end{itemize}

    \textbf{University of Manchester}, Manchester, United Kingdom \vspace{2mm}\\\vspace{1mm}%
    \textsl{Teaching Assistant} \hfill \textbf{September 2010 -- January 2011}\\
    Teaching assistant for undergraduate courses (Programming in C).


    %__________________________________________________________________________________________________________________
    % Education
    \section{\mysidestyle Education}

    \textbf{University of Manchester}, Manchester, United Kingdom \vspace{2mm}\\\vspace{1mm}%
    \textsl{Doctor of Philosophy} \hfill \textbf{ September 2009 -- June 2013}\vspace{-3mm}\\\vspace{-1mm}%
    \begin{list2}
        \item Thesis title: ``Measurements and Simulations of Impedance Reduction Techniques in Particle Accelerators" supervised by Professor Roger Jones and Dr Elias Metral (CERN)- Work done under placement at CERN.
        \item I used theoretical models, commercial EM modelling software (CST Microwave Studio and Ansoft HFSS) and bench top RF measurements to evaluate and characterise the EM properties of accelerator components. Working in an internationally diverse team, results were regularly presented internally for review and feedback, with communication between operational and design teams to integrate the RF results into their design revisions.
    \end{list2}\vspace{-1.5mm}
    \textbf{University of Southampton}, Southampton, United Kingdom \vspace{2mm}\\\vspace{1mm}% 
    \textsl{MPhys (Hons) Physics Classification: 1$^{\mathrm{st}}$ Class} \hfill \textbf{October 2005 -- July 2009}\vspace{-3mm}\\\vspace{-1mm}%
    \begin{list2}
        \item Masters' Thesis: ``Controlling the synthesis of branched gold nanoparticles using a wet chemical synthesis method'' resulting in a journal publication under the supervision of Dr. Antonios Kanaras
        \item Courses in experimental design, classical and quantum physics, mathematics, general relativity and fluid dynamics
    \end{list2}\vspace{-1.5mm}

    \textbf{King Edward VI Grammar School}, Chelmsford, United Kingdom  \vspace{2mm}\\\vspace{1mm}%
    \textsl{A Levels Physics, Mathematics, Chemistry, History} \hfill \textbf{2000 -- 2005}\vspace{-3mm}\\\vspace{-1mm}%
\vspace{-1.5mm}

  %__________________________________________________________________________________________________________________
    % Volunteer Experience
    \section{\mysidestyle Projects}


    \textbf{CosmicPi}  \hfill \textsl{https://www.cosmicpi.org}\\\vspace{1mm}%
    \textsl{Software and Physics Analysis} \hfill \textbf{June 2014 -- Present}\\
    \begin{itemize}
    \item{Building an low-cost, open source, distributed cosmic ray detector. Based on a RaspberryPi, a scintillator and custom electronics.}
    \item{Taking lead on physics capabilities and analysis tools. Adapting data for use with HiSPARC's Sapphire framework}
    \end{itemize}

    \textbf{OpenCosmics} \hfill \textsl{https://github.com/OpenCosmics/opencosmics.github.io}\\\vspace{1mm}%
    \begin{itemize}
    \item{Providing a framework to allow interoperable analysis of different cosmic ray experiments' data sets}
    \item{Building on HiSPARC's Sapphire framework, we're working to import different data sources in a functional manner}
    \end{itemize}

    %__________________________________________________________________________________________________________________
    % Volunteer Experience
    \section{\mysidestyle Volunteer\\Experience}


    \textbf{THE Port}, Geneva, Switzerland \vspace{2mm}\\\vspace{1mm}%
    \textsl{Fundraising, Event Organisation, Team Mentor} \hfill \textbf{June 2014 -- Present}\\
    \begin{itemize}
    \item{Organisation of a humanitarian hackathon in coordination with Red Cross, UN and other bodies}
    \item{Assisted in fundraising, recruiting volunteers for event and mentoring participants}
    \end{itemize}

    \textbf{CERN}, Geneva, Switzerland \vspace{2mm}\\\vspace{1mm}%
    \textsl{CERN Tour Guide/Conferencier} \hfill \textbf{November 2010 -- June 2016}\\
    \begin{itemize}
    \item{Tour guide to audience from secondary school students to visiting dignitaries}
    \item{Gave introductory talks about CERN and it's mission and tours to experimental areas}
    \end{itemize}

    \textbf{Pint of Science CH}, Geneva, United Kingdom \vspace{2mm}\\\vspace{1mm}%
    \textsl{Geneva Coordinator} \hfill \textbf{October 2013 -- May 2014}\\
    \begin{itemize}
    \item{Managed a city wide public science event serving over 250 members of the public in 2 pubs over 2 nights}
    \item{Recruited and coordinated venue managers, and liased with national organising committee}
    \end{itemize}

    \textbf{LHComedy/Comedy Collider}, Geneve, United Kingdom \vspace{2mm}\\\vspace{1mm}%
    \textsl{Advertisement/Performer} \hfill \textbf{July 2013 -- June 2014}\\
    \begin{itemize}
    \item{Responsible for advertising and marketing}
    \item{Stand up comedy events taking place at CERN in September 2013 and June 2014. starring professional physicists giving amateur stand up comedy sets alongside professional comedians to the general public in a sold out 200 seat capacity venue}
    \end{itemize}



    %__________________________________________________________________________________________________________________
    % Peer-Reviewed Publications
    \section{\mysidestyle Peer-Reviewed Publications}

    
	Day, H.A.,  Bartczak, D.,  Fairbairn, N.,  McGuire, E., Ardakan, M., Porter, A. E. and Kanaras, A.G.,
    ``Controlling the three-dimensional morphology of nanocrystals'',
    \textsl{CrystEngComm}, 2010, 12, 4312-4316.

\vspace{-2mm}


    %__________________________________________________________________________________________________________________
    % Conference Publications
    \section{\mysidestyle Conference Publications}

    
	Day, H.A., Barnes, M., Ducimetière, L. Vega Cid, L., Weterings, W.    ``Current and Future Beam Thermal Behaviour of the LHC Injection Kicker Magnet '',
    THPMW031, IPAC'16, Busan, Korea, 2016.

\vspace{-2mm}

	 Barnes, M.J., Adraktas, A., Bregliozzi, G., Calatroni, S., Day, H., Ducimetière, L., Goddard, B., Gomes Namora, V., Mertens, V., Salvant, B., Uythoven, J, Vega Cid, L., Weterings, W.,
Yin Vallgren, C. .    ``Operational Experience of the Upgraded LHC Injection Kicker Magnets '',
    THPMW033, IPAC'16, Busan, Korea, 2016.

\vspace{-2mm}

	 Barnes, M.J., Adraktas, A., Beck, M., Bregliozzi, G., Day, H., Ducimetière, L., Ferreira Somoza, J.A., Goddard, B., Krame, T., Pasquino, C., Rumolo, G., Salvant, B., Uythoven, J, Sermus, L. Vega Cid, L., Velotti, F., Weterings, W., Zannini, C.
Yin Vallgren, C. .    ``Studies of Impedance-related Improvements of the SPS Injection Kickers '',
    THPMW033, IPAC'16, Busan, Korea, 2016.

\vspace{-2mm}

	 Adraktas, A., Barnes, M.J., Day, H., Ducimetière, L. Yin Vallgren, C. .    ``High Voltage Performance of Surface Coatings on Alumina Insulators '',
    THPMW028, IPAC'16, Busan, Korea, 2016.

\vspace{-2mm}

	 B. Salvant, O. Aberle, M. Albert, R. Alemany-Fernandez, G. Arduini, J. Baechler, M.J. Barnes, P. Baudrenghien, O. E. Berrig, N. Biancacci, G. Bregliozzi, F. Carra, F. Caspers, P. Chiggiato, A. Danisi, H.A. Day, M. Deile, D. Druzhkin, S. Jakobsen, J. Kuczerowski, A. Lechner, R. Losito, A. Masi, E. Métral, N. Minafra, A. Nosych, A. Perillo Marcone, D. Perini, S. Redaelli, F. Roncarolo, G. Rumolo, E. Shaposhnikova, J. Uythoven, A. J. Välimaa, J. E. Varela Campelo, C. Vollinger, M. Wendt, J. Wenninger, C. Zannini, J.F. Esteban Muller. 
    ``Beam Induced RF Heating in the LHC in 2015 '',
    MOPOR008, IPAC'16, Busan, Korea, 2016.

\vspace{-2mm}

	 C. Belver-Aguilar, A. Faus-Golfe, F. Toral, M.J. Barnes, H. Day
    ``Transverse Impedance Measurements and DC Breakdown Tests on the First Stripline Kicker Prototype for the CLIC Damping Ring '',
    MOPOR008, IPAC'15, Richmond, USA, 2015.

\vspace{-2mm}

	H. Day, M.J. Barnes, L. Coralejo Feliciano
    ``Impedance Studies of the LHC Injection Kicker Magnets for HL-LHC'',
    MOPJE038, IPAC'15, Richmond, USA, 2015.

\vspace{-2mm}

	H. Day, M.J. Barnes, F. Caspers, E. Mètral, B. Salvant
    ``Beam Coupling Impedance of the New Beam Screen of the LHC Injection Kicker Magnets'',
    TUPRI030, IPAC'14, Dresden, Germany, 2014.

\vspace{-2mm}

	M.J. Barnes, P. Adraktas, G. Bregliozzi, S. Calatroni, P. Costa-Pinta, H. Day, L. Ducimetière, T. Kramer, V. Namora, V. Mertens, M. Taborelli
    ``High Voltage Performance of the Beam Screen of the LHC Injection Kicker Magnets'',
    MOPME074, IPAC'14, Dresden, Germany, 2014.

\vspace{-2mm}

	 C. Belver-Aguilar, A. Faus-Golfe, F. Toral, M.J. Barnes, H. Day
    ``Measurements and Laboratory Tests on a Prototype Stripline Kicker for the CLIC Damping Rings'',
    MOPRO027, IPAC'14, Dresden, Germany, 2014.

\vspace{-2mm}

	H. Day, M.J. Barnes, F. Caspers, E. Mètral, B. Salvant
    ``Evalutation of the Beam Coupling Impedance of the New Beam Screen Designs for the LHC Injection Kickers'',
    TUPME033, IPAC'13, Shanghai, China, 2013.

\vspace{-2mm}

	Day, H.A., Barnes, M., Caspers, F., Jones, R.M., Metral, E., Salvant, B.
    ``Evaluation of the Beam Coupling Impedance of New Beam Screen Designs for the LHC Injection Kicker Magnets'',
    WEPPR071, IPAC'12, New Orleans, US, 2012.

\vspace{-2mm}

	Day, H.A., Caspers, F., Dallocchio, A., Gentini, L., Grudiev, A., Jones, R.M., Metral, E., Salvant, B.
    ``Beam Coupling Impedance of the LHC TCTP Collimators'',
    WEPPR070, IPAC'12, New Orleans, US, 2012.

\vspace{-2mm}

	Day, H.A., Caspers, F., Jones, R.M., Metral, E.
    ``Simulations of Coaxial Wire Measurements of the Impedance of Asymmetric Structures. '',
    MOPS079, IPAC'11, San Sebastian, Spain, 2011.

\vspace{-2mm}

	Day, H.A., Barnes, M., Caspers, F., Jones, R.M., Metral, E., Salvant, B., C. Zannini
    ``Coaxial Wire Measurements of Ferrite Kicker Magnets'',
    MOPS078, IPAC'11, San Sebastian, Spain, 2011.

\vspace{-2mm}

	Day, H.A., Caspers, F., Jones, R.M., Metral, E., Salvant, B.
    ``Comparison of the current LHC Collimators and the SLAC Phase 2 Collimator Impedances'',
    MOPS080, IPAC'11, San Sebastian, Spain, 2011.

    %__________________________________________________________________________________________________________________
    % Invited Talks
    \section{\mysidestyle Invited Talks}

	Day, H.A., Biancacci, N., Salvant, B. Zannini, C.
    ``Impedance simulations of the LHC collimators and low beta simulations of ferrite kicker magnets with CST Particle Studio'',
    6th CST European User Group Meeting, Freising, Germany, 2011.

	Day, H.A.
    ``Impedance Reductin Techniques: LHC Collimators and Injection Kicker Magnets'',
    ICFA Mini-Workshop on Electromagnetic Wakefields and Impedances in Particle Accelerators, Erice, Italy, 2014.

	Day, H.A.,Barnes, M.J..
    ``Surface Flashover in Fast Pulsed Kicker Systems'',
    5th International Workshop on Mechanisms of Vacuum Arcs, Saariselka, Finland, 2015.


    %__________________________________________________________________________________________________________________
    % Language Skills
    \section{\mysidestyle Languages} 

   English (Fluent), French (Professional Proficiency), German (Beginner)



    %__________________________________________________________________________________________________________________
    % Referees
%    \section{\mysidestyle Referees} 
%    {\sl Available on request.}



%______________________________________________________________________________________________________________________
\section{\mysidestyle Referees} 

\begin{tabular}{@{}p{6cm}p{6cm}}
\textbf{Dr. Michael Barnes}  &   \textbf{Dr. Elias Metral}                       \\
CERN                     &  CERN                      \\
Geneve 23, Switzerland, CH-1211           &  Geneve 23, Switzerland, CH-1211        \\
phone: \textsl{available on request}    &  phone: \textsl{available on request}     \\
e-mail: \textsl{mike.barnes@cern.ch}   &  e-mail: \textsl{elias.metral@cern.ch}    \\
\end{tabular}


%______________________________________________________________________________________________________________________
\end{resume}
\end{document}


%______________________________________________________________________________________________________________________
% EOF

