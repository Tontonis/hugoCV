%------------------------------------
% Dario Taraborelli
% Typesetting your academic CV in LaTeX
%
% URL: http://nitens.org/taraborelli/cvtex
% DISCLAIMER: This template is provided for free and without any guarantee 
% that it will correctly compile on your system if you have a non-standard  
% configuration.
% Some rights reserved: http://creativecommons.org/licenses/by-sa/3.0/
%------------------------------------

%!TEX TS-program = xelatex
%!TEX encoding = UTF-8 Unicode

\documentclass[10pt, a4paper]{article}
\usepackage{fontspec} 

% DOCUMENT LAYOUT
\usepackage{geometry} 
\geometry{a4paper, textwidth=5.5in, textheight=8.5in, marginparsep=7pt, marginparwidth=.6in}
\setlength\parindent{0in}

% FONTS
\usepackage[usenames,dvipsnames]{color}
\usepackage{xunicode}
\usepackage{xltxtra}
\defaultfontfeatures{Mapping=tex-text}
\setromanfont [Ligatures={Common}, Numbers={OldStyle}, Variant=01]{Linux Libertine O}
\setmonofont[Scale=0.8]{Monaco}

% ---- CUSTOM COMMANDS
\chardef\&="E050
\newcommand{\html}[1]{\href{#1}{\scriptsize\textsc{[html]}}}
\newcommand{\pdf}[1]{\href{#1}{\scriptsize\textsc{[pdf]}}}
\newcommand{\doi}[1]{\href{#1}{\scriptsize\textsc{[doi]}}}
% ---- MARGIN YEARS
\usepackage{marginnote}
\newcommand{\amper{}}{\chardef\amper="E0BD }
\newcommand{\years}[1]{\marginnote{\scriptsize #1}}
\renewcommand*{\raggedleftmarginnote}{}
\setlength{\marginparsep}{7pt}
\reversemarginpar

% HEADINGS
\usepackage{sectsty} 
\usepackage[normalem]{ulem} 
\sectionfont{\mdseries\upshape\Large}
\subsectionfont{\mdseries\scshape\normalsize} 
\subsubsectionfont{\mdseries\upshape\large} 

% PDF SETUP
% ---- FILL IN HERE THE DOC TITLE AND AUTHOR
\usepackage[dvipdfm, bookmarks, colorlinks, breaklinks, 
% ---- FILL IN HERE THE TITLE AND AUTHOR
	pdftitle={Albert Einstein - vita},
	pdfauthor={My name},
	pdfproducer={http://nitens.org/taraborelli/cvtex}
]{hyperref}  
\hypersetup{linkcolor=blue,citecolor=blue,filecolor=black,urlcolor=MidnightBlue} 

% DOCUMENT
\begin{document}
{\LARGE Albert Einstein}\\[1cm]
 Institute for Advanced Study\\
Einstein Drive\\
Princeton, N.J. \texttt{08540}
U.S.A.\\[.2cm]
Phone: \texttt{609-734-8000}\\
Fax: \texttt{609-924-8399}\\[.2cm]
email: \href{mailto:a.einstein@ias.edu}{a.einstein@ias.edu}\\
\textsc{url}: \href{http://www.ias.edu/spfeatures/einstein/}{http://www.ias.edu/spfeatures/einstein/}\\ 
\vfill
 Born:  March 12, 1879---Ulm, Germany\\
Nationality:  German/American

%%\hrule
\section*{Current position}
\emph{Emeritus Professor}, Institute for Advanced Study, Princeton

%%\hrule
\section*{Areas of specialization}
 Physics • Relativity theory

%%\hrule
\section*{Appointments held}
\noindent
\years{1903-1908}Swiss Patent Office, Bern\\
\years{1908-1911}University of Bern\\
\years{1911-1912}University of Zürich\\
\years{1912-1914}Charles University of Prague\\
\years{1914-1932}Prussian Academy of Sciences, Berlin\\
\years{1920-1930}University of Leiden\\
\years{1932-1955}Institute for Advanced Study, Princeton

%\hrule
\section*{Education}
\noindent
\years{1900}\textsc{MSc} in Physics, ETH Zürich\\
\years{1900}\textsc{PhD} in Physics, ETH Zürich

%\hrule
\section*{Grants, honors \& awards}
\noindent
\years{1921}Nobel Prize in Physics, Nobel Foundation

\section*{Publications \& talks}

\subsection*{Journal articles}
\noindent
\years{1901}Einstein, Albert (1901), “Folgerungen aus den Capillaritätserscheinungen (Conclusions Drawn from the Phenomena of Capillarity)", \emph{Annalen der Physik} 4: 513\\
\years{1905a}Einstein, Albert (1905), “On a Heuristic Viewpoint Concerning the Production and Transformation of Light", \emph{Annalen der Physik} 17: 132–148.\\
\years{1905b}Einstein, Albert (1905), A new determination of molecular dimensions. \emph{PhD dissertation}.\\
\years{1905c}Einstein, Albert (1905), “On the Motion—Required by the Molecular Kinetic Theory of Heat—of Small Particles Suspended in a Stationary Liquid", \emph{Annalen der Physik} 17: 549–560. 
\years{1905d}Einstein, Albert (1905), “On the Electrodynamics of Moving Bodies", \emph{Annalen der Physik} 17: 891–921.\\
\years{1905e}Einstein, Albert (1905), “Does the Inertia of a Body Depend Upon Its Energy Content?", \emph{Annalen der Physik} 18: 639–641.\\
\years{1915}Einstein, Albert (1915), “Die Feldgleichungen der Gravitation (The Field Equations of Gravitation)", \emph{Koniglich Preussische Akademie der Wissenschaften}: 844–847\\
\years{1917a}Einstein, Albert (1917), “Kosmologische Betrachtungen zur allgemeinen Relativitätstheorie (Cosmological Considerations in the General Theory of Relativity)", \emph{Koniglich Preussische Akademie der Wissenschaften}\\
\years{1917b}Einstein, Albert (1917), “Zur Quantentheorie der Strahlung (On the Quantum Mechanics of Radiation)", \emph{Physikalische Zeitschrift} 18: 121–128

\subsection*{Books}
\noindent
\years{1954}Einstein, Albert (1954), \emph{Ideas and Opinions}, New York: Random House, ISBN 0-517-00393-7

\subsection*{Newspaper articles}
\noindent
\years{1940}Einstein, Albert, et al. (December 4, 1948), “To the editors", \emph{New York Times}\\
\years{1949}Einstein, Albert (May 1949), “Why Socialism?", \emph{Monthly Review}.

\section*{Teaching}

...

%\hrule
\section*{Service to the profession}

...
%\vspace{1cm}
\vfill{}
%\hrulefill

\begin{center}
{\scriptsize  Last updated: \today\- •\- 
% ---- PLEASE LEAVE THIS BACKLINK FOR ATTRIBUTION AS PER CC-LICENSE
Typeset in \href{http://nitens.org/taraborelli/cvtex}{
\fontspec{Times New Roman}\XeTeX }\\
% ---- FILL IN THE FULL URL TO YOUR CV HERE
\href{http://nitens.org/taraborelli/cvtex}{http://nitens.org/taraborelli/cvtex}}
\end{center}

\end{document}